\begin{resumo}


Este trabalho visa apresentar uma abordagem de um conversor tensão frequência que será usado numa tag de rádio frequência (\textit{Radio Frequency Identification}(RFID)) de monitoramento de sinais vitais vestível e alimentada passivamente. Usando diferentes bandas de radio frequência \textit{ultrahigh frequency} (UHF), \textit{ultrawideband} (UWB) os sinais vitais de pacientes podem ser captados e transmitidos com um sistema \textit{wireless}. As bandas de frequência UHF alimentam o circuito integrado (CI) e a transmissão é feita pela UWB que, com a variação da fonte de alimentação capta e transmite os sinais vitais.
Conforme o artigo base\cite{artigo_principal} um CI desse tipo pode ser alimentado com distâncias de até 51 metros (m) e transmitir as informações até 2 metros do receptor, além de ser um CI \textit{Ultra Low Power} consumindo uma potência próxima de 1$\mu$W (Watt). Por conta da complexidade do projeto, das pesquisas necessárias e a quantidade de  trabalho para a implementação desse CI, a atividade foi dividida em uma equipe de alunos projetistas e um professor que orientou e participou na finalização do projeto. Este documento se trata apenas de uma parte do CI, que é o conversor tensão frequência. Nele será apresentado os componentes e sub circuitos que compõem o sistema para a conversor bem como as pesquisas que levaram à elaboração desse sistema. Os resultados obtidos com as simulações serão analisados e interpretados.





% O resumo deve ressaltar o objetivo, o método, os resultados e as conclusões 
% do documento. A ordem e a extensão
% destes itens dependem do tipo de resumo (informativo ou indicativo) e do
% tratamento que cada item recebe no documento original. O resumo deve ser
% precedido da referência do documento, com exceção do resumo inserido no
% próprio documento. (\ldots) As palavras-chave devem figurar logo abaixo do
% resumo, antecedidas da expressão Palavras-chave:, separadas entre si por
% ponto e finalizadas também por ponto. O texto pode conter no mínimo 150 e 
% no máximo 500 palavras, é aconselhável que sejam utilizadas 200 palavras. 
% E não se separa o texto do resumo em parágrafos.

 \vspace{\onelineskip}
    
 \noindent
 \textbf{Palavras-chave}: Conversor tensão frequência, baixo consumo, monitoramento de sinais vitais.
\end{resumo}
