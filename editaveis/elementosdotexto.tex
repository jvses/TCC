%\part{Texto e Pós Texto}
\chapter[Considerações Finais]{Considerações Finais}

\section{Conclusões}

Nesse trabalho foram destrinchados os circuitos, topologias e sub-sistemas  para a projeção de um conversor tensão frequência além de simulações na ferramenta Cadence. 

Nas simulações os resultados obtidos se aproximaram das expectativas em ambas as três faixas de frequência. Em geral a frequência do VFC varia na faixa de 101,6Hz até 93,57kHz e a potência máxima obtida foi de 10,03$\mu$W.

Os blocos que compõem o circuito foram explicados em detalhes no texto e referenciados. Em geral os objetivos de baixo consumo e da faixa de frequência atingida foram satisfatórios. Entretanto há pontos que podem ser lapidados em trabalhos futuros afim de deixar o circuito mais robusto e resistente. 



\section{Trabalhos futuros}

Para trabalhos futuros as principais possibilidades de melhora que percebo para o projeto são os seguintes:
\begin{itemize}
\item Compactar AmpOps em um bloco
\item Vacinar o VIC contra a variação de temperatura
\end{itemize}

Incluindo as melhorias em trabalhos futuros poderia listá-los da seguinte forma:

\begin{enumerate}
\item \textbf{Compactar AmpOps em um bloco}: Ao fazer o símbolo do AmpOp e usá-lo em simulação ele considera cada um com a sua própria relação de transistores conectados em diodo. Então para não duplicar os divisores de tensão usados para tensões de bias eles podem ser feitos num esquemático só compartilhando a mesma relação de transistores. Assim a área usada no projeto de Layout será reduzida.
\item \textbf{Vacinar o VIC contra a variação de temperatura}: Toda a relação do VFC é influenciada pela natureza dos transistores que mudam o seu comportamento com a variação de temperatura. O circuito crucial para a conversão final em frequência é o VIC que alarga e eleva as faixas de frequência de saída com o aumentar da temperatura.
\item \textbf{Realizar a análise de Corners}: Considerando ou não essas mudanças, a análise de cornes aponta o quão suscetível a erros o circuito está com base na variação dos parâmetros que os componentes do circuito apresenta. Com essa análise é possível ver o comportamento do VFC em situações de piores cenários.
\item \textbf{Fazer Layout}: Projetar o layout do circuito para ser implementado na \textit{TAG}.   
\end{enumerate}

Para a realização desses trabalhos é preciso conversar com o professor orientador do projeto sobre os prazos para planejar um cronograma definitivo. Além de que as melhorias não são cruciais para o funcionamento do projeto, elas iram melhorar a sua performance. Já a análise de corners e Layout são essenciais e não devem ser dispensadas. Com isso em mente exibi na tabela \ref{tab:cronograma} uma sugestão de cronograma com um intervalo de tempo de uma semana para eu tentar implementar as melhorias sugeridas e o tempo para a continuação do projeto de layout e análise de corners.

\begin{table}[htb]
\centering
\begin{tabular}{|c|c|}
\hline 
Compactar AmpOps & 08/01/2024 a 15/01/2024 \\ 
\hline 
Vacinar o VIC & 16/01/2024 a 23/01/2024 \\ 
\hline 
Fazer análise de Corners & 24/01/2024 a 01/02/2024  \\ 
\hline 
Fazer Layout & 02/02/2024  a 15/03/2024\\ 
\hline 
\end{tabular} 
\caption{Sugestão de cronograma}
\label{tab:cronograma}
\end{table}






