\begin{resumo}[Abstract]
 \begin{otherlanguage*}{english}
This paper presents an approach for a voltage-to-frequency converter to be used in a wearable radio-frequency (RFID) tag, intended for passive monitoring of vital signs. Using different radio-frequency bands, such as \textit{ultrahigh frequency} (UHF) and \textit{ultrawideband} (UWB), it is possible to capture and transmit vital signs of patients through a \textit{wireless} system. The UHF band powers the integrated circuit (IC), while the UWB is responsible for transmitting vital signs, varying according to the power source. According to the base article \cite{artigo_principal}, this type of IC can be powered at a distance of up to 51 meters and transmit data up to 2 meters from the receiver, in addition to being an ultra-low-consumption IC, with power close to 1$\mu$W. Due to the complexity of the project and the necessary research, the work was carried out by a team of student designers, under the guidance of a professor. This document focuses on a specific part of the IC: the voltage-frequency converter. The components and subcircuits that make up the system will be presented, as well as the research that supported its development. Finally, the results of the simulations will be analyzed and interpreted.

   \vspace{\onelineskip}
 
   \noindent 
   \textbf{Key-words}: Voltage-frequency converter, radio frequency tag, low consumption, vital signs monitoring.
 \end{otherlanguage*}
\end{resumo}
